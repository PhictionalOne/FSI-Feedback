\documentclass[a4paper]{scrartcl}

\usepackage[utf8]{inputenc}
\usepackage{fancyhdr}
\usepackage{geometry}
\usepackage{fdsymbol}

%Größe der Ränder setzen
\geometry{a4paper,left=2cm, right=2cm, top=2cm, bottom=2cm}

% ----------------------- TODO ---------------------------
% Hier die wichtigen Infos eintragen

\newcommand{\VERANSTALTUNG}{FSI-Workshop: Git I}
\newcommand{\SPEAKER}{Alice Bob}
% Was soll auf die Rückseite gemalt werden?
\newcommand{\OBJECT}{eine Kartoffel}

% ---------------------- IGNORE --------------------------
\begin{document}
% Befehle für die Fragen
\newcommand{\question}[1]{
	\noindent
	\begin{tabular}{|p{\textwidth}|}
		\hline \textsf{#1} \\
		\hline \\[80pt]
		\hline
	\end{tabular} \\
}

\newcommand{\StudSem}{
	\begin{tabular}{|c|l|}
		\hline Studiengang & Fachsemester\\
		\hline & \\
		\hline
	\end{tabular}
}

% Kopf- und Fußzeile
\pagestyle{fancy}
\fancyhead[L]{\LARGE Feedback \VERANSTALTUNG \small \\ by \SPEAKER}
\fancyhead[C]{}
\fancyhead[R]{\StudSem}
\fancyfoot{}

% ----------------------- TODO ---------------------------
% Hier werden die Fragen eingetragen:

\question{Was hab ich verstanden?}\\\\
\question{Was hab ich nicht verstanden?}\\\\
\question{Was hat mir gut gefallen?}\\\\
\question{Was hat mir nicht gefallen?}\\\\
\question{Was ich dem Vortragenden sagen möchte?}\\


\noindent
\large Bewertung: $\largewhitestar \largewhitestar \largewhitestar \largewhitestar \largewhitestar$ \\\\
\noindent
Male auf die Rückseite \OBJECT:


\end{document}